\documentclass[border=2pt]{standalone}

% Circuit
\usepackage{circuitikz}

% Notation
\usepackage{siunitx}

\begin{document}
	
	\begin{circuitikz}
		% Specifications
		%% Resistors
		\ctikzset{resistors/zigs=5, resistors/width=1.5}
		%% Inductors
		\ctikzset{inductors/coils=6, quadpoles/transformer core/height=3.571, inductors/width=2}
		
		% Grid
%		\draw[help lines] (0,0) grid (14,13);
%		
%		\foreach \x in {0,1,2,3,...,13}
%		{
%			\node at (\x,-0.5) {\x};
%			\node at (-0.5, \x) {\x};
%		}		

		
		% Nodes
		\node at (0,2.5) {\SI{65}{V} AC};
		\node at (6.5,4.3) {$P_1$};
		
		% Circuit
		%% Left Part
		
		\draw (7,0) -- (0,0) to[short, -*] +(0,2);
		\draw (0,3) to[short, *-] +(0,2) -- (1.2,5) to[pR, wiper pos=0.5, name=A] (1.2,0);			
		\draw (A.wiper) -- (2,2.5) -- (2,5) to[rmeterwa, t=\si{A}, l=$I_1$] (4.5,5) -- +(0.5,0) to [rmeterwa, t=\si{W}] (7,5)
		 node[american, transformer core, anchor=A1] (T) {}
		 (T.A1) node {}
  		 (T.A2) node {}
      	 (T.B1) node {} 
      	 (T.B2) node {}
      	 (T.base) node {};
		\draw (5.1,5) -- (5.1,6) -- (6,6) -- (6,5.42);
		\draw (6,4.58) -- (6,0);
      	%% Right Part		
		\draw (4.5,5) to[rmeterwa, t=\si{V}, l=$V_1$] +(0,-5);
		\draw (9,5) -- (9.5,5) to[rmeterwa, t=\si{A}, l=$I_2$] (12,5) -- (12,3) -- (12.5,3);
		\draw (9.5,5) to[rmeterwa, t=\si{V}, l=$V_2$] (9.5,0); 
		\draw (9,0) -- (13,0) to[pR, -*, wiper pos=0.5, name=B, l_=$R_L$] (13,6);
	\end{circuitikz}

	
\end{document}