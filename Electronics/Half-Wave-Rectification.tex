\documentclass[border=3pt]{standalone}

% Circuits
\usepackage[european,s traightvoltages, RPvoltages, americanresistor, americaninductors]{circuitikz}
\tikzset{every picture/.style={line width=0.2mm}}

% Notation
\usepackage{amsmath}

% Tikz Library
\usetikzlibrary{calc}

% Bipoles Specifications
\ctikzset{bipoles/thickness=1.2, label distance=1mm, voltage shift = 1}

% Inductors
\ctikzset{inductors/coils=4, inductors/width=1.2}

% Transformers
\ctikzset{quadpoles/transformer core/height=1.8}

\begin{document}
	\begin{circuitikz}
%		%Grid
%		\def\length{6}
%		\draw[thin, dotted] (-\length,-\length) grid (\length,\length);
%		\foreach \i in {1,...,\length}
%		{
%			\node at (\i,-2ex) {\i};
%			\node at (-\i,-2ex) {-\i};	
%		}
%		\foreach \i in {1,...,\length}
%		{
%			\node at (-2ex,\i) {\i};	
%			\node at (-2ex,-\i) {-\i};	
%		}
%		\node at (-2ex,-2ex) {0};
		
		%Circuit
		\def\x{4}
		\def\y{3}
		\draw 
		(0,0) node[transformer core] (T) {}
		(T.B1) to[full diode, l=$D_1$] ++(\x,0) 
				to[R, l=$R$] ($(T.B2)+(\x,0)$) -- ++(-0.8*\x,0) 
				node[ground, pos=0.5] (ground) {} |- (T-L2.midtap);
				
		\draw[fill=black] (ground.north) circle (1.5pt);
	\end{circuitikz}
\end{document}