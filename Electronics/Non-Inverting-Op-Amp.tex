\documentclass[border=3pt]{standalone}

% Circuits
\usepackage[european,s traightvoltages, RPvoltages, americanresistor, americaninductors]{circuitikz}
\tikzset{every picture/.style={line width=0.2mm}}

% Notation
\usepackage{amsmath}

% Tikz Library
\usetikzlibrary{calc}

% Bipoles Specifications
\ctikzset{bipoles/thickness=1.2, label distance=1mm, voltage shift = 1}

\begin{document}

	\begin{circuitikz}

%		%Grid
%		\def\length{4}
%		\draw[thin, dotted] (-\length,-\length) grid (\length,\length);
%		\foreach \i in {1,...,\length}
%		{
%			\node at (\i,-2ex) {\i};
%			\node at (-\i,-2ex) {-\i};	
%		}
%		\foreach \i in {1,...,\length}
%		{
%			\node at (-2ex,\i) {\i};	
%			\node at (-2ex,-\i) {-\i};	
%		}
%		\node at (-2ex,-2ex) {0};
		
		%Circuit
		\node[op amp, noinv input up] at (0,0) (opamp) {};
		\node[ground] at (-3.19,-2.5) {};
		\draw (opamp.-) -- ++(-0.5,0) -- ++(0,-1.15) to[short,-*] ++(-1.5,0);
		\draw (opamp.+) -- ++(-2,0) to[sV, l_=$v_\text{IN}$] ++(0,-2) -- ++(0,-1);
		\draw[-latex] (opamp.up) -- ++(0,1) node[above] {$V_+$};
		\draw[-latex] (opamp.down) -- ++(0,-1) node[below] {$V_-$};
		\draw (opamp.out) to[short, -*] ++(1,0) node[shift={(0.6,0)}] {$v_\text{O}$};

	\end{circuitikz}
	
\end{document}