\documentclass[border=3pt]{standalone}

% Circuits
\usepackage[european,s traightvoltages, RPvoltages, americanresistor, americaninductors]{circuitikz}
\tikzset{every picture/.style={line width=0.2mm}}

% Notation
\usepackage{amsmath}

% Tikz Library
\usetikzlibrary{calc}

% Bipoles Specifications
\ctikzset{bipoles/thickness=1.2, label distance=1mm, voltage shift = 1}

% Arrows Above Compenents
% Source: https://tex.stackexchange.com/questions/574576/circuitikz-straight-voltage-arrows-with-fixed-length
\newcommand{\fixedvlen}[3][0.4cm]{% [semilength]{node}{label}
    % get the center of the standard arrow
    \coordinate (#2-Vcenter) at ($(#2-Vfrom)!0.5!(#2-Vto)$);
    % draw an arrow of a fixed size around that center and on the same line
    \draw[-{Triangle[round,open]}] ($(#2-Vcenter)!#1!(#2-Vfrom)$) -- ($(#2-Vcenter)!#1!(#2-Vto)$);
    % position the label as it where if standard voltages were used
    \node[ anchor=\ctikzgetanchor{#2}{Vlab}] at (#2-Vlab) {#3};
}
\newcommand{\fixedvlendashed}[3][0.75cm]{% [semilength]{node}{label}
    % get the center of the standard arrow
    \coordinate (#2-Vcenter) at ($(#2-Vfrom)!0.5!(#2-Vto)$);
    % draw an arrow of a fixed size around that center and on the same line
    \draw[dashed,-{Triangle[round,open]}] ($(#2-Vcenter)!#1!(#2-Vfrom)$) -- ($(#2-Vcenter)!#1!(#2-Vto)$);
    % position the label as it where if standard voltages were used
    \node[ anchor=\ctikzgetanchor{#2}{Vlab}] at (#2-Vlab) {#3};
}

\begin{document}
	
	\begin{circuitikz}
%		%Grid
%		\def\length{6}
%		\draw[thin, dotted] (-\length,-\length) grid (\length,\length);
%		\foreach \i in {1,...,\length}
%		{
%			\node at (\i,-2ex) {\i};
%			\node at (-\i,-2ex) {-\i};	
%		}
%		\foreach \i in {1,...,\length}
%		{
%			\node at (-2ex,\i) {\i};	
%			\node at (-2ex,-\i) {-\i};	
%		}
%		\node at (-2ex,-2ex) {0};
		
		% Circuit
		\def\y{4}
		\def\x{3}
		\draw (0,0) 
			to[battery1, v^>, name=v_in] ++(0,\y) -- ++(\x,0) 
			to[full diode, l_=$D$, v^<, name=v_d, i_>=$I$] ++(0,-\y*0.5) 
			to[R, l_=$R$, v^<, name=v_r] ++(0,-\y*0.5) -- ++(-\x,0);
		\draw (1.5,0) to[short,*-] ++(0,-0.1) node[ground] {};
		
		% Voltages
		\fixedvlen[0.4cm]{v_in}{$V_\text{IN}$}
		\fixedvlen[0.4cm]{v_d}{$V_D$}
		\fixedvlen[0.4cm]{v_r}{$V_R$}
			
	\end{circuitikz}
	
\end{document}